%!TEX program = xelatex
\documentclass[UTF8]{ctexart}

% math bracket
%  ()
\newcommand{\brc}[1]{\left({{}#1}\right)}
%  []
\newcommand{\brm}[1]{\left[{{}#1}\right]}
%  ||
\newcommand{\brv}[1]{\left|{{}#1}\right|}
%  {}
\newcommand{\brf}[1]{\left\{{{}#1}\right\}}
%  ||
\newcommand{\brt}[1]{\left\Vert{{}#1}\right\Vert}
%  <>
\newcommand{\brg}[1]{\left<{{}#1}\right>}
%  floor
\newcommand{\floor}[1]{\lfloor{{}#1}\rfloor}
%  ceil
\newcommand{\ceil}[1]{\lceil{{}#1}\rceil}

% font
\newcommand{\fira}[1]{{\firacode {}#1}}

% abbr command
\newcommand{\ds}{\displaystyle}
\newcommand{\pt}{\partial}
\newcommand{\rint}[2]{\Big|^{{}#1}_{{}#2}}
\newcommand{\leg}{\left\lgroup}
\newcommand{\rig}{\right\rgroup}

% math symbol
\newcommand{\de}{\mathrm{d}}
\newcommand{\im}{\mathrm{im}}
\newcommand{\ord}{\mathrm{ord}}
\newcommand{\cov}{\mathrm{Cov}}
\newcommand{\lub}{\mathrm{LUB}}
\newcommand{\glb}{\mathrm{GLB}}
\newcommand{\var}{\mathrm{Var}}
\newcommand{\aut}{\mathrm{Aut}}
\newcommand{\sylow}{\mathrm{Sylow}}
\newcommand{\xhi}{\mathcal{X}}
\newcommand{\po}{\mathcal{P}}
\newcommand{\bi}{\mathrm{b}}
\newcommand{\rfl}{\mathcal{R}}

% algorithmic symbol
\newcommand{\gro}{\mathrm{O}}

\newfontfamily\firacode{Fira Code}
\newfontfamily\mincho{MS Mincho}

% math
\usepackage{ntheorem}
\usepackage{ulem}

\theoremseparator{ }
\newtheorem{dft}{Definition}
\newtheorem{tem}[dft]{Theorem}
\newtheorem{lem}[dft]{Lemma}
\newtheorem{epe}[dft]{Example}
\newtheorem{cor}[dft]{Corollary}

\usepackage{mathrsfs}
\usepackage{amsmath}
\usepackage{amssymb}
\usepackage{cancel}
%\usepackage{amsthm}

% control
\usepackage{ifthen}
\usepackage{intcalc}

% format
\usepackage{indentfirst}
\usepackage{enumerate}
\usepackage{url}
\usepackage{setspace}
\usepackage[utf8]{inputenc}  

\usepackage{xcolor}

\definecolor{ballblue}{rgb}{0.13, 0.67, 0.8}
\definecolor{celestialblue}{rgb}{0.29, 0.59, 0.82}
\definecolor{bananayellow}{rgb}{1.0, 0.88, 0.21}
\definecolor{brilliantlavender}{rgb}{0.96, 0.73, 1.0}
\definecolor{burgundy}{rgb}{0.5, 0.0, 0.13}
\definecolor{cadmiumorange}{rgb}{0.93, 0.53, 0.18}
\definecolor{aqua}{rgb}{0.0, 1.0, 1.0}
\definecolor{auburn}{rgb}{0.43, 0.21, 0.1}
\definecolor{brass}{rgb}{0.71, 0.65, 0.26}
\definecolor{tangerine}{rgb}{0.95, 0.52, 0.0}
\definecolor{portlandorange}{rgb}{1.0, 0.35, 0.21}
\definecolor{mediumred-violet}{rgb}{0.73, 0.2, 0.52}

% listing set
\definecolor{func}{rgb}{0.29, 0.59, 0.82}
\definecolor{ftype}{rgb}{0.95, 0.52, 0.0}
\definecolor{cls}{rgb}{1.0, 0.35, 0.21}
\definecolor{slf}{rgb}{0.73, 0.2, 0.52}
\definecolor{backg}{HTML}{F7F7F7}
\definecolor{str}{HTML}{228B22}
% attestation
\definecolor{atte}{RGB}{178,34,34}

\usepackage{listings}

\lstset{
    backgroundcolor = \color{backg},
    basicstyle = \small,%
    escapeinside = ``,%
    keywordstyle = \color{func},% \underbar,%
    identifierstyle = {},%
    commentstyle = \color{blue},%
    stringstyle = \color{str}\ttfamily,%
    %labelstyle = \tiny,%
    extendedchars = false,%
    linewidth = \textwidth,%
}

\usepackage{geometry}

\geometry{
    left=2.0cm,
    right=2.0cm,
    top=2.5cm,
    bottom=2.5cm
}

\usepackage[
    colorlinks,
    linkcolor=blue,
    anchorcolor=blue,
    citecolor=blue
]{hyperref}

% table
\usepackage{csvsimple}
\usepackage{longtable}
\usepackage{dcolumn}

% graph
\usepackage{tikz}
\usepackage{pgfplots}
\usetikzlibrary{
    quotes,
    angles,
    matrix,
    arrows,
    automata
}


\title{盲点钻石矿工}
\author{Myriad Dreamin 2017211279 2017211301}
\date{}

\pgfplotsset{compat=1.16}
\begin{document}
\setlength{\parindent}{2em}
%\renewcommand{\baselinestretch}{1.5}
\setlength{\baselineskip}{2.5em}
\maketitle
\tableofcontents

\newpage

\section{约定}
\subsection{mp,n,m}
本文出现的$\mathrm{mp}$,总是四连通图模型。$n$表示$\mathrm{mp}$的行数,$m$表示$\mathrm{mp}$的列数。

\subsection{dp}
对于贪心法的dp,有二维.第一维表示行号,第二维保存列号。

\subsection{k阶子式}
矩阵$A$,第$i_1,i_2,\dots, i_k$行,第$j_1,j_2,\dots, j_k$列交叉形成的子式的表示法为:
$$
    A\begin{pmatrix}
        i_1;i_2;\dots; i_k\\
        j_1;j_2;\dots; j_k\\
    \end{pmatrix}
$$
\section{程序控制}
本次算法测试在Windows10平台,c++规范为g++/c++11.\\
\indent 本次算法测试框架为上次排序算法的测试框架的抽象化,接受一个测试匿名函数和断言匿名函数.框架会在进入测试函数的时候计时,在离开测试函数的时候停止计时。如果定义了宏信号\fira{DEBUG},并且传入了断言函数,框架会调用断言函数.\\
\indent 算法时间测试的粒度为$1\mu s$.\\
\indent 压入匿名函数的效率很高,并且是强制无参匿名函数,所以压函数栈和退函数栈的时间短于$1\mu s$,可以将这段时间忽略不计.\\
\indent 此次算法的主要矛盾在于,如何获得最优值,而不是算法效率,因此算法时间复杂度在可以接受的范围内即可。本文也几乎只在意最终获取的总价值。\\
\indent 考虑降低算法输入的规模.如果地图过大,显然可以将多个格点合并为一个格点,则降低了输入规模。 
\section{c++生成二维正态分布}
因为带有$\rho$参数的二维正态分布的难度和代价比较高,所以本文只使用$\vec X = (x_1,x_2)$两变量独立的正态分布.\\
\indent \fira{functory\_range($\sigma_l$,$\sigma_r$)}接受两个方差值,返回一个调制函数\fira{functory},这个函数返回一个函数,函数中包含的两个正态的方差$\sigma\in [\sigma_l, \sigma_r]$.\\
\indent \fira{functory}接受一个二维点$\vec \mu = (\mu_1,\mu_2)$,返回一个表达$(\mu_1,\mu_2)$服从正态分布的随机向量的函数\fira{generator}.\\
\indent \fira{generator}不接受输入,每次调用返回一个总体服从在($\mu_1$,$\mu_2$)上各自服从($\sigma_l$,$\sigma_r$)的向量.
\section{已知 $\{\mathrm{Manhattan}(\vec x, \vec y)\leqslant K\}$分布}
\subsection{贪心法}
贪心法原理:\\
\indent 根据动态规划的递推公式。当视野为$K$时,对于当前格点$F=(a,b)$,Manhattan距离$d<K$的格点是已知的,显然我们有:
$$
	\mathrm{dp}[x][y] = \max{\mathrm{dp}[x][y+1], \mathrm{dp}[x+1][y]} + \mathrm{mp}[x][y].
$$
\indent 其中$\mathrm{mp}[x][y]$表示真实矿产分布.\\
\indent 如果$\mathrm{dp}[a+1][b]>\mathrm{dp}[a][b]$,说明往下(往行号增大方向)走,在当前视野来看,是更优的。否则说明往右(往列号增大方向)走更优。\\
\indent 特殊情况的处理:
\begin{enumerate}[1]
	\item 如果$a=n,b\neq m$,说明已经走到地图最下侧,此时不能往下走.
	\item 如果$b=m,a\neq n$,说明已经走到地图最右侧,此时不能往右走.
	\item 如果$a=n,b=m$,说明已经结束,此时退出算法.
	\item 如果$k=1$,说明只有当前所在位置的视野,无法进行贪心.
\end{enumerate}

\subsection{猜图策略}
\indent 如何根据已有信息,猜测图剩下的矿产分布。这里有一个很简单的边缘衰减模糊化的方法。\\
\indent 假设矿工现在已经处于$(a,b)$,那么:
$$
	\mathrm{newmp}[a'][b'] =\left\{
        \begin{array}{ll}
            \mathrm{mp}[a'][b'] &, \mathrm{Manhattan}((a,b), (a',b')) \leqslant K;\\
            c\cdot (\mathrm{newmp}[a'-1][b'] +\mathrm{newmp}[a'][b'-1]), & , \mathrm{Manhattan}((a,b), (a',b')) > K.
        \end{array}
    \right.
$$
\subsubsection{c的选取}
对于$(a,b)$的真实值为$d$,它对$(a+1, b+1)$的贡献为$c(cd+cd)=2c^2d < d$.所以$0\leqslant c\leqslant \dfrac{\sqrt{2}}{2}$.否则会使衰减失败,对地图估计错误
\subsubsection{特点}
\begin{enumerate}[1]
	\item 当c=0时,模糊化消失,此时模糊化的贪心原则退化为原始贪心原则.
	\item 显然这种方法只能估计已经接触到的矿产风貌,对于完全隐藏在视野外的矿产,模糊化方式估计完全失效。
\end{enumerate}
\begin{table}[h]
    \centering
    \csvautotabular{miner.csv}
    \caption{贪心K步和猜图迭代,部分1}
\end{table}
\begin{table}[h]
    \centering
    \csvautotabular{miner_2.csv}
    \caption{贪心K步和猜图迭代,部分2}
\end{table}
\newpage
\ \\
\newpage
\section{缺失 c\%分布}
随机c\%个像素已知矿产含量,未知的地方填零。问该地图我们能得到多少价值?
\subsection{朴素情况}
不作任何处理,直接进行dp.这个方法已经较好,此时在$90\%\sim95\%$以内的未知矿产就能获得很大价值。

\subsection{图像推断的思想}
通过已知部分的图像还原整个图像信息,是图像补全的思想。现在已经是一个AI的热门领域,并且有过大量研究。当图像信息过少,补全难以做到,这个解决方案进一步变为图像推断。但是个人能力有限,从事的领域也不属于AI和计算领域,所以这里的图像推断采用另一种原始的方法——卷积模糊化做部分推断。
\subsubsection{卷积方式}
\indent 下文介绍的都是$3x3$卷积核,没有使用其他库,而是做了一个的简单版本使用。\\
\indent 设需要处理的矩阵为$A_{m\times n}$,将其扩充为:
$$
    E(A) = \begin{pmatrix}
        M_1 & M_2 & M_3\\
        M_4 & A & M_5\\
        M_6 & M_7 & M_8\\
    \end{pmatrix}
$$
这样即使是$A$的边缘元素也能做一般化处理。\\
\indent 设卷积核矩阵函数为$K_{3\times 3}: \mathcal{K}^3 \to \mathcal{K}$,则处理过后的矩阵即为
$$
    A'_{m\times n} = \brm{K(E(A)\begin{pmatrix}
        a(i)-1;a(i);a(i)+1\\
        a(j)-1;a(j);a(j)+1\\
    \end{pmatrix})}_{m\times n}
$$
\indent 其中$a$将原$A$矩阵的位置映向$E(A)$矩阵中对应的位置.\\
这里假设A矩阵都已经做了零值填充,即:
$$
    E_0(A) = \begin{pmatrix}
        0 & \mathbf{0}^{T} & 0\\
        \mathbf{0} & A & \mathbf{0}\\
        0 & \mathbf{0}^T & 0\\
    \end{pmatrix}
$$
\subsubsection{常滤波}
常滤波的核如下:
$$
    \mathrm{Ker}_{c} = \frac{1}{9}\begin{pmatrix}
        0 & 0 & 0\\
        0 & 1 & 0\\
        0 & 0 & 0\\
    \end{pmatrix}
$$
这个核没有任何作用,相当于卷积的单位元.
$$
    K_{c} = M(2;2).
$$
\subsubsection{均值滤波}
均值滤波的核如下:
$$
    \mathrm{Ker}_{ave} = \frac{1}{9}\begin{pmatrix}
        1 & 1 & 1\\
        1 & 1 & 1\\
        1 & 1 & 1\\
    \end{pmatrix}
$$
均值滤波的特性就仅仅是将附近值收集到一起。
$$
    K_{ave} = \sum_{i,j} \mathrm{Ker}_{ave}(i;j)M(i;j)
$$
\subsubsection{高斯滤波}
高斯滤波的核如下:
$$
    \mathrm{Ker}_{gauss} = \frac{1}{16}\begin{pmatrix}
        1 & 2 & 1\\
        2 & 4 & 2\\
        1 & 2 & 1\\
    \end{pmatrix}
$$
高斯滤波对不同的距离的值做了加权,本质还是一种线性滤波。
$$
    K_{gauss} = \sum_{i,j} \mathrm{Ker}_{gauss}(i;j)M(i;j)
$$
\subsubsection{中值滤波}
中值滤波是一种非线性滤波,返回矩阵九个值的中值元素。
$$
    K_{median} = \mathrm{Median}\{M(i;j)\}
$$
\newpage
\subsection{模糊化不影响判断}
% \begin{table}[ht]
%     \centering
%     \begin{tabular}{@{}p{1pt}@{}*6{c|}c}
%         \csvreader[
%             no head,
%             late after line=\\,
%             ]{ave_ave_absent_test_mutiple_1_10_guess2times_1_4.csv}{}{%
%             & \csvcoli & \csvcolii & \csvcoliii & \csvcoliv & \csvcolv & \csvcolvi & \csvcolvii 
%                 }%
%     \end{tabular}
% \end{table}
下面是$K_{ave}$作用于矩阵以后,dp与最优答案的比值表。
\begin{table}[h]
    \centering
\csvautotabular[]{ave_ave_absent_test_mutiple_1_10_guess2times_1_4.csv}
\caption{$\sigma \in [1,2]$, c=0.9}
\end{table}\\
\begin{table}[h]
    \centering
    \csvautotabular{ave_ave_absent_test_mutiple_1_10_guess2times_1_4_2.csv}
    \caption{$\sigma \in [1,2]$, c=0.9}
\end{table}\\
\indent 可以看见随着$K_{ave}$系数增大,改变价值的能力几乎是不变的。
\newpage
下面是$K_{gauss}$作用于矩阵以后,dp与与最优答案的比值表。
\begin{table}[h]
    \centering
\csvautotabular{gauss_gauss_absent_test_mutiple_1_10_guess2times_1_400.csv}
\caption{$\sigma \in [1,20]$, c=0.9}
\end{table}\\
\begin{table}[h]
    \centering
    \csvautotabular{gauss_gauss_absent_test_mutiple_1_10_guess2times_1_400_2.csv}
    \caption{$\sigma \in [1,20]$, c=0.9}
\end{table}\\
其与$K_{ave}$相同,对系数均不敏感。

\newpage
\subsection{价值集中程度对滤波估计的影响}
下表对比了$K_{ave},K_{gauss},K_{median}$受矿产分布方差的影响.
\begin{table}[h]
    \centering
    \csvautotabular{sigma_between_4_100_mapsize_40_40_guess2times.csv}
    \caption{scalar = $40\times 40$, $\sigma \in [\sqrt{s},2\sqrt{s}]$, c=0.9}
\end{table}\\
随着方差增大,除了$K_{median}$,$K_{c},K_{ave}, K_{gauss}$的效率下降.\\
而$K_{median}$的效率是出人意料的,猜想其效率上升的原因是。当$A$中孤立元素过多,$K_{median}$滤去了太多细节,因此获得的价值不如不优化。当$\sigma$增大时,$K_{median}$迅速补全缺失的聚集空隙,起到了还原的效果。
\newpage
\subsection{地图规模对滤波估计的影响}
下表对比了$K_{ave},K_{gauss},K_{median}$受地图规模的影响.
\begin{table}[h]
    \centering
    \csvautotabular{dense_between_100_10000_guess2times.csv}
    \caption{scalar = $[100, 10000]$, $\sigma = \sqrt{10}$, c=0.9}
\end{table}\\
选取了较为适中的$\sigma$,在这个环境下,明确看见,地图规模越大,估计越差。
\subsubsection{对$K_{gauss},K_{ave}$的进一步测试}
辛辛苦苦写出了$K_{gauss},K_{ave}$,看见其效果这么差.心有不甘,但是做过许多测验,最终都没能将这两个滤波器拯救回来.仔细想想,3x3核的作用实在有限,看来算法不是一味地生搬硬套。
\section{总结}
文章第一部分针对$k$步矿工问题提供了两个策略,一个贪心策略,一个猜图策略。贪心策略已经有比较好的表现,而猜图策略能够略微增加价值收益。\\
\indent 文章第二部分针对$c\%$矿工问题测试了基于几个滤波器的猜图策略。对于均值滤波器和高斯滤波器,并未有较好的表现,可能原因是滤波器影响范围太小,对图像补全没有很好的效果。\\
\indent 值得关注的是中值滤波器,它在有限的范围内(地图规模较小,或地图矿产分布较为分散)极大地提高了价值获取的效率。但是更加值得注意的一点是,即使有90\%的地图不知道,$K_c$仍然有较好的效果,这时候应该果断根据已有信息取得价值,而不是歪曲信息想方设法完全还原地图。\\
\indent 本文所有代码实现不依赖除C++标准库Algorithm中的有限简单函数,和不值得重复实现地无关紧要的函数以外其他的任何库,均有算法实现。
\end{document}